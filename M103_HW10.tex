\documentclass[11pt]{article}
\usepackage{graphicx}
\usepackage{amsthm}
\usepackage{amsmath}
\usepackage{amssymb}
\usepackage[shortlabels]{enumitem}
\usepackage[margin=1in]{geometry}

\newcommand{\C}{\mathbb{C}}
\newcommand{\Sum}{\sum\limits_{n=0}^{\infty}}
\newenvironment{solution}
  {\renewcommand\qedsymbol{$\blacksquare$}\begin{proof}[Solution]}
  {\end{proof}}

\setlength\parindent{0pt}

\newtheorem*{observation}{Observation}
\newtheorem*{theorem}{Theorem}
\newtheorem*{claim}{Claim}
\newtheorem*{corollary}{Corollary}

\theoremstyle{definition}
\newtheorem*{definition}{Definition}

\begin{document}

	\hrule
	\begin{center}
        \textbf{MATH103: Complex Analysis}\hfill \textbf{Fall 2023}\newline


		{\Large Homework 10}

		David Yang
	\end{center}

\hrule

\vspace{1em}


\textit{Chapter VIII (The Logarithmic Integral) Problems.} \\

\underline{Section VIII.4 (Open Mapping and Inverse Function Theorems), Problem 1}\\

\textbf{Suppose $D$ is a bounded domain with piecewise smooth boundary. Let $f(z)$ be meromorphic and $g(z)$ analytic on $D$. Suppose that both $f(z)$ and $g(z)$ extend analytically
across the boundary of $D$, and that $f(z) \neq 0$ on $\partial D$. Show that}
\[ \frac{1}{2\pi i} \oint_{\partial D} g(z) \frac{f'(z)}{f(z)} \, dz = \sum \limits_{j=1}^n m_j g(z_j) \]

\textbf{where $z_1, \dots, z_n$ are the zeros and poles of $f(z)$ and $m_j$ is the order of $f(z)$ at $z_j$.}

\begin{solution} Note that $g(z) \frac{f'(z)}{f(z)}$ is analytic on $D \, \cup \, \partial D$ except for a finite number of isolated singularities at $z_1, \dots, z_n$. Consequently, by the Residue Theorem, we have that
\[ \oint_{\partial D} g(z) \frac{f'(z)}{f(z)} \, dz = 2\pi i \sum\limits_{j=1}^n \mathrm{Res}\left[g(z) \frac{f'(z)}{f(z)}, z_j\right].\]

Consider a given singularity $z_j$, which is either a zero or pole of order $m_j$ at $f(z)$. By definition, we have that
\[ f(z) = (z-z_j)^{m_j}h(z)\]

for a function $h(z)$ satisfying $h(z_j) \neq 0$ and $h(z)$ analytic at $z_j$. By the Chain Rule, we also find that \[ f'(z) = m_j (z-z_j)^{m_j - 1} h(z) + (z-z_j)^{m_j} h'(z)\] and so 
\begin{align*} \frac{f'(z)}{f(z)} &= \frac{m_j (z-z_j)^{m_j - 1} h(z) + (z-z_j)^{m_j} h'(z)}{(z-z_j)^{m_j}h(z)} \\
&= \frac{m_j}{z-z_j} + \frac{h'(z)}{h(z)}.
\end{align*}

Since $h(z_j) \neq 0$ and $h(z)$ is analytic at $z_j$, $\frac{h'(z)}{h(z)}$ is also analytic at $z_j$. Thus, the residue of $\frac{f'(z)}{f(z)}$ at $z_j$, which is the coefficient of the $\frac{1}{z-z_j}$ term in the Laurent expansion about $z_j$, is simply $m_j$. \\

Plugging this residue into our result from the Residue Theorem, we find that
\begin{align*} \oint_{\partial D} g(z) \frac{f'(z)}{f(z)} \, dz &= 2\pi i \sum\limits_{j=1}^n \mathrm{Res}\left[g(z) \frac{f'(z)}{f(z)}, z_j\right] \\
&= 2\pi i \sum\limits_{j=1}^n m_jg(z_j)\end{align*}

where the factor of $g(z_j)$ follows from the fact that $g(z)$ is analytic at $z_j$. \\

Dividing both sides of our equation by $2\pi i$, we arrive at the desired result,

\[ \frac{1}{2\pi i} \oint_{\partial D} g(z) \frac{f'(z)}{f(z)} \, dz = \sum \limits_{j=1}^n m_j g(z_j) \]

where $z_1, \dots, z_n$ are the zeros and poles of $f(z)$ and $m_j$ is the order of $f(z)$ at $z_j$. \end{solution}
\newpage

\underline{Section VIII.6 (Winding Numbers), Problem 6}\\

\textbf{Let $\gamma$ be a closed path in a domain $D$ such that $W(, \gamma, \xi) = 0$ for all $\xi \notin D.$ 
Suppose that $f(z)$ is analytic on $D$ except possibly at finite number of isolated singularities $z_1, \dots, z_m \in D \setminus \Gamma$. Show that}

\[ \int_{\gamma} f(z) \, dz = 2\pi i \sum\limits_{k=1}^m W(\gamma, z_k) \mathrm{Res}[f, z_k]. \]

\begin{solution}
Note that $f(z)$ is analytic on $D$ except at its isolated singularities $z_1$ to $z_m$. Consequently, the function \[g(z) = f(z) - \sum\limits_{k=1}^m \sum\limits_{n=-\infty}^{-1} a_{n, \, k}(z-z_k)^n\] obtained by subtracting the principal parts of $f(z)$ at each singularity from $f(z)$ is analytic both at each singularity and everywhere else on $D$. Equivalently, $g(z)$ is analytic everywhere on $D$. \\

Since $g(z) = f(z) - \sum\limits_{k=1}^m \sum\limits_{n=-\infty}^{-1} a_{n, \, k}(z-z_k)^n$ is analytic on $D$, and for the closed path $\gamma$ in $D$, $W(\gamma, \xi) = 0$ for all $\xi \notin D$, we know by the Theorem on page 243 that
\[ \int_\gamma g(z) \, dz = \int_\gamma \left(f(z) - \sum\limits_{k=1}^m \sum\limits_{n=-\infty}^{-1} a_{n, \, k}(z-z_k)^n\right) \, dz = 0.\]

Rearranging the equation $\int_\gamma \left(f(z) - \sum\limits_{k=1}^m \sum\limits_{n=-\infty}^{-1} a_{n, \, k}(z-z_k)^n\right) \, dz = 0$, we find that
\begin{align*} \int_\gamma f(z) \, dz &= \int_\gamma \sum\limits_{k=1}^m \sum\limits_{n=-\infty}^{-1} a_{n, \, k}(z-z_k)^n \, dz \\
&= \sum\limits_{k=1}^m \int_\gamma \sum\limits_{n=-\infty}^{-1} a_{n, \, k}(z-z_k)^n \, dz\end{align*}

By VIII.6 Problem 5, we know that for each $n \leq -2$ and each singularity $z_k$ (since no singularity is on the trace of $\gamma$), $\int_\gamma (z-z_k)^n \, dz = 0$. Thus,  

\begin{align*} \int_\gamma f(z) \, dz
&= \sum\limits_{k=1}^m \int_\gamma \sum\limits_{n=-\infty}^{-1} a_{n, \, k}(z-z_k)^n \, dz \\
&= \sum\limits_{k=1}^m \int_\gamma \left(\sum\limits_{n=-\infty}^{-2} a_{n, \, k}(z-z_k)^n\right) + \left(a_{-1, \, k}(z-z_k)^{-1}\right) dz \\
&= \sum\limits_{k=1}^m \int_\gamma a_{-1, \, k} \frac{1}{z-z_k}\, dz \\
&= \sum\limits_{k=1}^m a_{-1, \, k} \int_\gamma \frac{1}{z-z_k}\, dz.\end{align*}

Note that $W(\gamma, z_k) = \frac{1}{2\pi i} \int_\gamma \frac{1}{z-z_k}\, dz$ and $\mathrm{Res}[f, z_k] = a_{-1, \, k}$ by definition, and so substituting these expressions to our current equation gives us

\[\int_\gamma f(z) \, dz = 2\pi i \sum_{k=1}^m W(\gamma, z_k)\mathrm{Res}[f, z_k] \]

where $z_1$ to $z_m$ are the isolated singularities of $f(z)$, as desired.\end{solution}
\end{document}