\documentclass[11pt]{article}
\usepackage{graphicx}
\usepackage{amsthm}
\usepackage{amsmath}
\usepackage{amssymb}
\usepackage[shortlabels]{enumitem}
\usepackage[margin=1in]{geometry}

\newcommand{\C}{\mathbb{C}}

\newenvironment{solution}
  {\renewcommand\qedsymbol{$\blacksquare$}\begin{proof}[Solution]}
  {\end{proof}}

\setlength\parindent{0pt}

\newtheorem*{observation}{Observation}
\newtheorem*{theorem}{Theorem}
\newtheorem*{claim}{Claim}
\newtheorem*{corollary}{Corollary}

\theoremstyle{definition}
\newtheorem*{definition}{Definition}

\begin{document}

	\hrule
	\begin{center}
        \textbf{MATH103: Complex Analysis}\hfill \textbf{Fall 2023}\newline

		{\Large Homework 2}

		David Yang
	\end{center}

\hrule

\vspace{1em}

\textit{Chapter II (Analytic Functions) Problems.} \\

\underline{Section II.1 (Review of Basic Analysis), II.1.14} \\

\textbf{Let $h(t)$ be a continuous complex-valued function on the unit interval $[0, 1]$, and consider} \[ H(z) = \int_0^1 \frac{h(t)}{t-z} \, dt. \]
\textbf{Where is $H(z)$ defined? Where is $H(z)$ continuous? Justify your asnwer. \textit{Hint}. Use the fact that if $|f(t) - g(t)| < \epsilon$ for $0 \leq t \leq 1$, then $\int_0^1 |f(t) - g(t)| \, dt  < \epsilon.$}\\

\begin{solution}
$H(z) = \int_0^1 \frac{h(t)}{t-z}$ is defined only when the integrand is defined; this happens only when the denominator of the fraction $\frac{h(t)}{t-z}$ is nonzero. Put simply, we need $t - z \neq 0$ or $z \neq t$. Since by definition $t \in [0, 1]$, $H(z)$ is defined for $z \in \C \setminus [0, 1].$ \\

We claim that $H(z)$ is continuous for all $z \in \C \setminus [0, 1]$ (by definition, it can only be continuous where it is defined, and so we aim to show that $H(z)$ is continuous at all points where it is defined). To do so, we will appeal to the limit definition of continuity, that $H(z)$ is continuous at $z_0$ if \[ \lim\limits_{z \rightarrow z_o} H(z) = H(z_0), \]

To make use of the hint, let us define $f(t) = \frac{h(t)}{t-z}$ and $g(t) = \frac{h(t)}{t-z_0}$ for any $z, z_0 \in \C \setminus [0, 1]$. Then
\[ |f(t) - g(t) |= \left| \frac{h(t)}{t-z} - \frac{h(t)}{t-z_0} \right| = \left| \frac{h(t)(z-z_0)}{(t-z)(t-z_0)}\right|\]

Since $h(t)$ is defined on the compact interval $[0, 1]$, it has a maximum value, which we will denote $M$. Equivalently, $h(t) \leq M$ for all $t \in [0, 1]$. Thus, substituting this back into our above equation and using the fact that $|ab| = |a||b|$, we get

\begin{align*}  |f(t) - g(t)| = \left| \frac{h(t)(z-z_0)}{(t-z)(t-z_0)}\right| &< \left| \frac{M(z-z_0)}{(t-z)(t-z_0)}\right|. \\
&= \left| \frac{(z-z_0)}{(t-z)(t-z_0)}\right| |M|\end{align*}

We claim that \[ \lim\limits_{z \rightarrow z_0} \left(\left| \frac{(z-z_0)}{(t-z)(t-z_0)}\right| |M| \right) = 0. \]

To see this, note that as $z \rightarrow z_0$, the denominator $(t-z)(t-z_0)$ approaches $(t-z_0)(t-z_0) = (t-z_0)^2$. Thus, rewriting the above limit, we have

\[ \lim\limits_{z \rightarrow z_0} \left(\left| \frac{(z-z_0)}{(t-z)(t-z_0)}\right| |M| \right) =  \lim\limits_{z \rightarrow z_0} \left(\left| \frac{(z-z_0)}{(t-z_0)^2}\right| |M| \right) . \]

Note that since by definition, $z_0 \notin [0, 1]$, $z_0$ cannot get arbitrarily close to $t$. On the other hand, the numerator $z-z_0$ tends towards $0$ as $z$ approaches $z_0$. Thus, 

\[  \lim\limits_{z \rightarrow z_0} \left(\left| \frac{(z-z_0)}{(t-z_0)^2}\right| |M| \right) = 0.\]

By the hint, we know that since $|f(t) - g(t)| < \epsilon$ for $0 \leq t \leq 1$, then $\int_0^1 |f(t) - g(t)| \, dt  < \epsilon.$ 

Equivalently, \[ \lim\limits_{z \rightarrow z_0}\int_0^1 \left| \frac{h(t)}{t-z} - \frac{h(t)}{t-z_0} \right| = 0. \]

Furthermore, note that by an absolute value property of integrals, we know that
\begin{align*}
    \int_0^1 \left| \frac{h(t)}{t-z} - \frac{h(t)}{t-z_0} \right| &\geq \left| \int_0^1 \frac{h(t)}{t-z} - \frac{h(t)}{t-z_0}\right| \\
    &= \left| \int_0^1 \frac{h(t)}{t-z} - \int_0^1 \frac{h(t)}{t-z_0} \right| \\
    &= |H(z) - H(z_0)|
\end{align*}

Put succinctly, we know that \[ H(z) - H(z_0) \leq \int_0^1 \left| \frac{h(t)}{t-z} - \frac{h(t)}{t-z_0} \right|. \]
Thus, since $\lim\limits_{z \rightarrow z_0}\int_0^1 \left| \frac{h(t)}{t-z} - \frac{h(t)}{t-z_0} \right| = 0$, we know that 
\[ \lim\limits_{z \rightarrow z_0} |H(z) - H(z_0)| = 0.\]

for any $z_0 \in \C \setminus [0, 1]$ (where $H$ is defined). Thus, by the limit definition of continuity, $H(z)$ is continuous everywhere it is defined.
\end{solution}

\newpage

\underline{Section II.3 (The Cauchy-Riemann Equations), II.3.4} \\

\textbf{Show that if $f$ is analytic on a domain $D$ , and if $|f|$ is constant, then $f$ is constant. \textit{Hint}. Write $\overline{f} = |f|^2/f.$}

\begin{solution}
We will split our work into two cases: if either $f$ is $0$ anywhere in the domain or if $f \neq 0$ everywhere in the domain. Note that by construction, these cover all possible cases for $f$.\\

First, if $f$ is zero anywhere in the domain, 
then $|f| = 0$ at that point. Since $|f|$ is constant, we know $|f|$ is zero for every point in the domain, which only occurs when $f$ is zero everywhere. Since $f$ is zero everywhere in this domain, then $f$ is constant, as desired. \\

On the other hand, if $f \neq 0$ everywhere in the domain, then we consider $\overline{f}$. By the hint, we know 
\[ \overline{f} = \frac{|f|^2}{f} = \frac{C}{f} \]

for some constant $C$, since $|f|$ is constant. Furthermore, note that $f$ is analytic on $D$. 
Since the quotient of an analytic function is also analytic (when the denominator does not vanish -- which it does not since , $f$ is $0$ nowhere in the domain), we know $\overline{f}$ is also analytic on $D$. \\

By Exercises II.3.3, since both $f$ and $\overline{f}$ are analytic on $D$, $f$ is constant. \\

Thus, by our two cases, we know that if $f$ is analytic on a domain $D$ , and if $|f|$ is constant, then $f$ is constant. \end{solution}

\newpage

\underline{Section II.4 (Inverse Mappings and the Jacobian), II.4.3} \\

\textbf{Consider the branch of $f(z) = \sqrt{z(1-z)}$ on $\C \setminus [0, 1]$ that has positive imaginary part at $z=2$. 
What is $f'(z)$? Be sure to specify the branch of the expression for $f'(z)$.} \\

\begin{solution}

    We can calculate $f'(z)$ using the Chain Rule: note that $f(z) = (z(1-z))^{\frac{1}{2}}$ so
\begin{align*}
     f'(z) &= \frac{1}{2} (z(1-z))^{-\frac{1}{2}} \cdot (z(1-z))' \\
     &= \frac{1}{2} (z(1-z))^{-\frac{1}{2}} \cdot (1-2z).
\end{align*}

Simplifying, we get that \[ f'(z) = \frac{1-2z}{2\sqrt{z(1-z)}}. \]

Multiplying both the numerator and denominator of $f'(z)$ by $\sqrt{z(1-z)}$, we get that
\begin{align*}
    f'(z) &= \frac{1-2z}{2\sqrt{z(1-z)}} \cdot \frac{\sqrt{z(1-z)}}{\sqrt{z(1-z)}} \\
    &= \frac{(1-2z)\sqrt{z(1-z)}}{2z(1-z)}
\end{align*}

To determine the branch of the expression for $f'(z)$, we can first analyze the branch of $f(z)$ at $z=2$. By definition, we know
\begin{align*}
    f(z) = z\sqrt{1-z} &= z(1-z)^{\frac{1}{2}} \\
    &= ze^{\frac{1}{2} \left(\log |1-z| + i \mathrm{Arg}(1-z) + i\cdot 2\pi m \right)}.
\end{align*}

At $z=2$, $\log |1-z| = 0$ and $\mathrm{Arg}(1-z) = \mathrm{Arg}(-1) = -\pi$, so we have
\[ z\sqrt{1-z} = 2e^{\frac{1}{2} \left(i \pi + i 2\pi m \right)} = 2e^{i \frac{\pi}{2}}e^{i \pi m}. \]

Since we are considering the branch of $f(z)$ on $\C \setminus [0, 1]$ has positive imaginary part at $z=2$, we are considering the principal branch where $m = 0$. \\

For $f'(z)$, notice that the expression $\sqrt{z(1-z)} = f(z)$ appears in the numerator; consequently, the branch of the expression for $f'(z)$ is simply the same principal branch of $f(z)$ which has positive imaginary part at $z=2$. 
\end{solution}

\end{document}
