\documentclass[11pt]{article}
\usepackage{graphicx}
\usepackage{amsthm}
\usepackage{amsmath}
\usepackage{amssymb}
\usepackage[shortlabels]{enumitem}
\usepackage[margin=1in]{geometry}

\newcommand{\C}{\mathbb{C}}
\newcommand{\Sum}{\sum\limits_{n=0}^{\infty}}
\newenvironment{solution}
  {\renewcommand\qedsymbol{$\blacksquare$}\begin{proof}[Solution]}
  {\end{proof}}

\setlength\parindent{0pt}

\newtheorem*{observation}{Observation}
\newtheorem*{theorem}{Theorem}
\newtheorem*{claim}{Claim}
\newtheorem*{corollary}{Corollary}

\theoremstyle{definition}
\newtheorem*{definition}{Definition}

\begin{document}

	\hrule
	\begin{center}
        \textbf{MATH103: Complex Analysis}\hfill \textbf{Fall 2023}\newline


		{\Large Homework 8}

		David Yang
	\end{center}

\hrule

\vspace{1em}


\textit{Chapter VI (Laurent Series and Isolated Singularities) and Chapter VII (The Residue Calculus) Problems.} \\

\underline{Section VI.2 (Isolated Singularities of an Analytic Function), Problem 12}\\

\textbf{Show that if $z_0$ is an isolated singularity of $f(z)$ that is not removable, then $z_0$ is an essential singularity of $e^{f(z)}$.}

\begin{solution} First, note that if $z_0$ is an isolated singularity of $f(z)$, it must also be an isolated singularity of $e^{f(z)}$.\footnote{this follows from the definition of an isolated singularity -- we can take the same radius $r$ that $f(z)$ is analytic in the punctured disk with radius $r$ and center $z_0$ in and the same property will hold for $e^{f(z)}$.} 
Since $z_0$ is an isolated singularity of $f(z)$ that is not removable, we know that $z_0$ is either an essential singularity of $f(z)$ or $z_0$ is a pole of $f(z)$. We will consider these two cases separately.\\

First, suppose that $z_0$ is an essential isolated singularity of $f(z)$. Then by the Casorati-Weierstrass Theorem, we know that for every complex number $w_0$, there exists a sequence $z_n \rightarrow z_0$ such that $f(z_n) \rightarrow w_0$. \\

Consider two sequence $a_n \rightarrow z_0$ and $b_n \rightarrow z_0$ such that $f(a_n) \rightarrow 0$ and $f(b_n) \rightarrow x$, for any complex $x$. Then we must have that
\[ \left|e^{f(a_n)}\right| \rightarrow e^0 = 1 \text{ and } \left|e^{f(b_n)} \right| \rightarrow e^{x}.\]

Since $\left|e^{f(a_n)}\right| \rightarrow e^0 = 1$ for a sequence $a_n \rightarrow z_0$, we know that $z_0$ cannot be a pole for $e^{f(z)}$ (since the magnitude of $e^{f(z)}$ does not approach $\infty$ as $a_n \rightarrow z_0$). Similarly, since $\left|e^{f(b_n)} \right| \rightarrow e^{x}$ and the choice of $x$ is arbitrary, we know that
$e^{f(z)}$ is not bounded near $z_0$, and thus, $z_0$ cannot be a removable singularity of $e^{f(z)}$. \\

On the other hand, suppose that $z_0$ is a pole of $f(z)$, say of order $N$ at $z_0$. By definition, then, we know that \[ f(z) = \frac{g(z)}{(z-z_0)^N} = \frac{h(z)}{(z-z_0)^N} + r(z),\]

where the first statement follows from the definition of a pole of $f(z)$ at order $N$ at $z_0$, with $g(z)$ analytic at $z_0$ and $g(z_0) \neq 0$.
The second statement follows from the fact that we can rewrite $f(z)$ as its Laurent decomposition (where the first term represents the principal part at the pole $z_0$, and the second term $r(z)$ is analytic). Furthermore, note that under this definition, $h(z)$ is a polynomial of degree $<N$ with $h(z_0) \neq 0$. \\

Now, consider 
\[ e^{f(z)} = e^{\frac{h(z)}{(z-z_0)^N} + r(z)} = e^{r(z)} e^{\frac{h(z)}{(z-z_0)^N}}. \]

Note that the Laurent expansion of $e^{r(z)}$, about $z_0$, includes no negative powers of $(z-z_0)$, since $r(z)$ is analytic at $z_0$. On the other hand, the Laurent expansion of $e^{\frac{h(z)}{(z-z_0)^N}}$ about $z_0$ is \[e^{\frac{h(z)}{(z-z_0)^N}} = \Sum \frac{\left( \frac{h(z)}{(z-z_0)^N}\right)^n}{n!}. \]

Since $h(z_0) \neq 0$ by construction and $h(z)$ is a polynomial of degree $<N$, we know that for each fixed $n$, there is a term with a negative power. Since this is an infinite sum, and these terms will not cancel with the resulting terms for larger values of $n$, there must be infinitely many negative power terms in the Laurent expansion about $z_0$. 
Thus, by definition, $e^{f(z)}$ has an essential singularity at $z_0$. \\

In both cases, we find that $e^{f(z)}$ has an essential singularity at $z_0$, and thus, if $z_0$ is an isolated singularity of $f(z)$ that is not removable, then $z_0$ is an essential singularity of $e^{f(z)}$, as desired. \end{solution}

\newpage

\underline{Section VII.1 (The Residue Theorme), Problem 2}\\

\textbf{Calculate the residue at each isolated singularity in the complex plane of the following functions.}

\begin{enumerate}[a)]
	\item $e^{1/z}$
	
	\begin{solution}
		Note that the isolated singularity of $e^{1/z}$ is at $z=0$. The Laurent Series of $e^{1/z}$ at $z=0$ is 
		\[ e^{1/z} = 1 + \frac{1}{z} + \frac{1}{2!}\frac{1}{z^2} + \dots \]

		By definition, the residue of $e^{1/z}$ at the isolated singularity $z=0$ is the coefficient of $\frac{1}{z}$ in the Laurent expansion at $z=0$, so \[ \mathrm{Res}\left[ e^{1/z}, 0\right] = \boxed{1}.\]
	\end{solution}

	\item $\tan z$
	
	\begin{solution}
		Note that the isolated singularities of $\tan z = \frac{\sin z}{\cos z}$ occur when $\cos z = 0$, so $z=\frac{\pi}{2} + n\pi$ for any $n \in \mathbb{Z}$. \\

		Since $\cos z$ has a simple zero at each of these singularities (they are simple since its derivative, $-\sin z$, is nonzero at the singularities), and both $\cos z$ and $\sin z$ are analytic at the singularities, then we know by Rule 3 that
		\[ \mathrm{Res}\left[ \frac{\sin z}{\cos z}, s_n\right] = \frac{\sin s_n}{-\sin s_n} = \boxed{-1}\] for each singularity $s_n = \frac{\pi}{2} + n\pi$.\end{solution}

	\item $\frac{z}{(z^2+1)^2}$
	
	\begin{solution}
	Note that the isolated singularities of $\frac{z}{(z^2+1)^2}$ occur when $z^2+1=0$, so $z = \pm i.$ Furthermore, since \[\frac{1}{\left(\frac{z}{(z^2+1)^2}\right)} = \frac{(z^2+1)^2}{z} = \frac{(z+i)^2(z-i)^2}{z} \]
	has zeros of order $2$ at the singularities $z = \pm i$, we know that $\frac{z}{(z^2+1)^2}$ has double poles at $\pm i$.  \\

	By Rule 2, we have that
	\begin{align*} \mathrm{Res}\left[\frac{z}{(z^2+1)^2}, i\right] &= \lim\limits_{z \rightarrow i} \frac{d}{dz} \left[ (z-i)^2 \frac{z}{z^2+1}\right] \\
	&= \lim\limits_{z \rightarrow i} \frac{d}{dz} \left[ \frac{z}{(z+i)^2}\right] \\
	&= \lim\limits_{z \rightarrow i} \frac{-z + i}{(z+i)^3} \\
	&= 0.\end{align*}

	Similarly, by Rule 2, we have that

	\begin{align*} \mathrm{Res}\left[\frac{z}{(z^2+1)^2}, -i\right] &= \lim\limits_{z \rightarrow -i} \frac{d}{dz} \left[ (z+i)^2 \frac{z}{z^2+1}\right] \\
		&= \lim\limits_{z \rightarrow -i} \frac{d}{dz} \left[ \frac{z}{(z-i)^2}\right] \\
		&= \lim\limits_{z \rightarrow -i} -\frac{z + i}{(z-i)^3} \\
		&= 0.\end{align*}	
	
	Thus, we conclude that $\boxed{ \mathrm{Res}\left[\frac{z}{(z^2+1)^2}, i\right] = 0 \text{ and } \mathrm{Res}\left[\frac{z}{(z^2+1)^2}, -i\right] = 0 }$.
	\end{solution}

	\item $\frac{1}{z^2+z}$
	
	\begin{solution}
	Note that the isolated singularities of $\frac{1}{z^2+z} = \frac{1}{z(z+1)}$ occur at $z = 0, -1$. Furthermore, note that
	\[ \frac{1}{z^2+z} = \frac{1}{z(z+1)} = \frac{1}{z} - \frac{1}{z+1}. \]
	
	Note that the Laurent expansion of this expression at $z=0$ is 
	\[ \frac{1}{z} - \frac{1}{z+1} = \frac{1}{z} - \frac{1}{1 - (-z)} = \frac{1}{z} + \Sum (-z)^k.\]

	By definition, the residue at the isolated singularity $z=0$ is the coefficient of $\frac{1}{z}$ in the Laurent expansion at $z=0$, so \[ \mathrm{Res}\left[ \frac{1}{z^2+z}, 0\right] = 1.\]
	
	By a similar line of reasoning, note that the Laurent expansion of this expression at $z=-1$ is 
	\[ \frac{1}{z} - \frac{1}{z+1} = \frac{-1}{1-(z+1)} - \frac{1}{z+1} = -\frac{1}{z+1} - \Sum (z+1)^k.\]

	By definition, the residue at the isolated singularity $z=-1$ is the coefficient of $\frac{1}{z+1}$ in the Laurent expansion at $z=-1$, so \[ \mathrm{Res}\left[ \frac{1}{z^2+z}, -1\right] = -1.\]

	Thus, we find that $\boxed{\mathrm{Res}\left[ \frac{1}{z^2+z}, 0\right] = 1 \text{ and } \mathrm{Res}\left[ \frac{1}{z^2+z}, -1\right] = -1}.$
\end{solution}
\end{enumerate}

\end{document}
