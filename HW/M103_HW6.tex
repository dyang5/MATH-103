\documentclass[11pt]{article}
\usepackage{graphicx}
\usepackage{amsthm}
\usepackage{amsmath}
\usepackage{amssymb}
\usepackage[shortlabels]{enumitem}
\usepackage[margin=1in]{geometry}

\newcommand{\C}{\mathbb{C}}
\newcommand{\Sum}{\sum\limits_{n=0}^{\infty}}
\newenvironment{solution}
  {\renewcommand\qedsymbol{$\blacksquare$}\begin{proof}[Solution]}
  {\end{proof}}

\setlength\parindent{0pt}

\newtheorem*{observation}{Observation}
\newtheorem*{theorem}{Theorem}
\newtheorem*{claim}{Claim}
\newtheorem*{corollary}{Corollary}

\theoremstyle{definition}
\newtheorem*{definition}{Definition}

\begin{document}

	\hrule
	\begin{center}
        \textbf{MATH103: Complex Analysis}\hfill \textbf{Fall 2023}\newline


		{\Large Homework 6}

		David Yang
	\end{center}

\hrule

\vspace{1em}


\textit{Chapter V (Power Series) Problems.} \\

\underline{Section V.4 (Power Series Expansion of an Analytic Function), Problem 4} \\

\textbf{Suppose $f(z)$ is analytic at $z = 0$ and satisfies $f(z) = z + f(z)^2$. What is the radius of convergence of the power series expansion of $f(z)$ about $z=0$?} 

\begin{solution}
For $f(z)$ to satisfy $f(z) = z + f(z)^2$, it must also satisfy \[f(z)^2 - f(z) + z =0.\]

Solving this equation for $f(z)$ using the Quadratic Formula, we find that
\[ f(z) = \frac{1 \pm \sqrt{1-4z}}{2}.\]

Let $g(z)$ be the solution (one of $\frac{1 + \sqrt{1-4z}}{2}$ and $\frac{1 - \sqrt{1-4z}}{2}$) that satisfies $f(0) = g(0)$. By definition, both $f(z)$ and $g(z)$ satisfy $f(z) = z + f(z)^2$ and $g(z) = z+g(z)^2$, so taking the derivatives, we find that
\[ 0 = 2f(z)f'(z) - f'(z) + 1 \text{ and } 0 = 2g(z)g'(z) - g'(z) + 1.\]

Solving for $f'(z)$ and $g'(z)$, we get that
\[ f'(z) = \frac{1}{1 - 2f(z)} \text{ and } g'(z) = \frac{1}{1 - 2g(z)}.\]

Since by construction, $f(0) = g(0)$, we must also have that $f'(0) = g'(0)$, and we can follow this same process to conclude that $f^{(n)}(0) = g^{(n)}(0)$ for any positive integer $n$. \\

Thus, since the derivatives of $f$ and $g$ are the same at $z=0$, their power series are the same and they must have the same radius of convergence. To determine the radius of convergence
of the power series expansion of $f(z)$ about $z=0$, then, we can simply determine the radius of convergence of $g(z)$ about $z=0$. \\

Note that \[g'(z) = \frac{\mp 1}{\sqrt{1-4z}}\]
where the $\mp$ corresponds to the fact that $g(z) = \frac{1 + \sqrt{1-4z}}{2}$ or $\frac{1 - \sqrt{1-4z}}{2}$, depending on the value of $f(0)$. In either case, note that the derivative is not defined
at $z = \frac{1}{4}$. Thus, since the radius of convergence is the distance to the nearest singularity (from $z=0$), we conclude that the radius of convergence of the power series expansion of $f(z)$ about $z=0$ is $\boxed{\frac{1}{4}}.$\end{solution}

\underline{Section V.5 (Power Series Expansion at Infinity), Problem 4} \\

\textbf{Let $E$ be a bounded subset of the complex plane $\C$ over which area integrals can be defined, and set}
\[ f(w) = \iint_{E} \frac{dx \, dy}{w-z}, \, \, \, w \in \C \setminus E \]
\textbf{where $z = x+iy$. Show that $f(w)$ is analytic at $\infty$, and find a formula for the coefficients of the power series of $f(w)$ at 
$\infty$ in descending powers of $w$.}

\begin{solution}
To show that $f(w)$ is analytic at $\infty$, we can show that $g(w) = f\left(\frac{1}{w}\right)$ is analytic at $w = 0$. By definition,
\begin{align*} g(w) &= f\left(\frac{1}{w}\right) = \iint_E \frac{1}{\frac{1}{w} - z} \, dx \, dy \\
&= \iint_E \frac{w}{1-zw} \, dx \, dy.\end{align*}

Note that $\frac{w}{1-zw}$ is analytic at $w = 0$, since $\frac{d}{dw} \left(\frac{w}{1-zw}\right) = \frac{(1-wz) - w(1-z)}{(1-wz)^2}$ is continuous at $w=0$. Thus, since the integrand is analytic, we know that $g(w)$ is analytic at $w=0$. 
Equivalently, $f(w)$ is analytic at $\infty$ as desired. 

\noindent\rule{\textwidth}{1pt} \\

To find a formula for the coefficients of the power series of $f(w)$ at $\infty$ in descending powers of $w$, we will begin by rewriting the integrand in the form
of a geometric series sum: note that
\begin{align*} f(w) &= \iint_{E} \frac{1}{w-z} \, dx \, dy \\
&= \iint_{E} \frac{\frac{1}{w}}{1 - \frac{z}{w}} \, dx \, dy. \end{align*}

Experessing the integrand as the sum of a geometric series, we get that 
\[ f(w) = \iint_{E} \, \Sum \left(\frac{1}{w}\right)\left(\frac{z}{w}\right)^n \, dx \, dy. \]

Since the integral and sum can be interchanged, we find that this is equivalent to
\begin{align*} f(w) &= \Sum \iint_{E} \, \left(\frac{1}{w}\right)\left(\frac{z}{w}\right)^n \, dx \, dy \\
&= \Sum \left(\iint_{E} \, z^n \, dx \, dy \right)\frac{1}{w} \left(\frac{1}{w}\right)^n \\
&= \Sum \left(\iint_{E} \, z^n \, dx \, dy \right) \frac{1}{w^{n+1}}.  \end{align*}

Thus, the coefficient of $\frac{1}{w^{n+1}}$ in the power series expansion of $f(w)$ at $\infty$ is $\boxed{\iint_E z_n \, dx \, dy}$.
\end{solution}

\end{document}
