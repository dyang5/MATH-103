\documentclass[11pt]{article}
\usepackage{graphicx}
\usepackage{amsthm}
\usepackage{amsmath}
\usepackage{amssymb}
\usepackage[shortlabels]{enumitem}
\usepackage[margin=1in]{geometry}

\newcommand{\C}{\mathbb{C}}
\newcommand{\Sum}{\sum\limits_{n=0}^{\infty}}
\newenvironment{solution}
  {\renewcommand\qedsymbol{$\blacksquare$}\begin{proof}[Solution]}
  {\end{proof}}

\setlength\parindent{0pt}

\newtheorem*{observation}{Observation}
\newtheorem*{theorem}{Theorem}
\newtheorem*{claim}{Claim}
\newtheorem*{corollary}{Corollary}

\theoremstyle{definition}
\newtheorem*{definition}{Definition}

\begin{document}

	\hrule
	\begin{center}
        \textbf{MATH103: Complex Analysis}\hfill \textbf{Fall 2023}\newline


		{\Large Homework 7}

		David Yang
	\end{center}

\hrule

\vspace{1em}


\textit{Chapter V (Power Series) Problems.} \\


\underline{Section V.7 (The Zeros of an Analytic Function), Problem 11}\\

\textbf{Show that if $f(z)$ is a nonconstant analytic function on a domain $D$, then the image under $f(z)$ of any open set is open.
\textit{Remark}. This is the open mapping theorem for analytic functions. The proof is easy when $f'(z) \neq 0$, since the Jacobian of $f(z)$ coincides with $|f'(z)|^2$. Use Exercise $9$ to deal with the points where $f'(z)$ is zero.}

\begin{solution}
Let $S$ be an open set in $D$, and let $V$ be the image of $S$ under $f$. Consider any $w \in V$. By definition, there exists some $z_0 \in S$ such that $f(z_0) = w$. \\

We will separate our work into two cases, when $f'(z_0) \neq 0$ and when $f'(z_0) = 0$. Let us begin with the first case, when $f'(z_0) \neq 0$. Since $f$ is analytic in $D$, $z_0 \in S \subset D$, and $f'(z_0) \neq 0$, we know by the Inverse Function Theorem that there is a 
disk of radius $r$ centered at $z_0$, which we can denote $D_r(z_0)$, such that $f(D_r(z_0))$ is open. By definition, since $D_r(z_0) \subset S$, $f(D_r(z_0)) \subset V$. Consequently, we know there is a disk around $w$ that is fully contained in $V$ for any arbitrary $w \in V$, and thus, by definition, $V$ is open. We conclude that 
the image of $f(z)$ of any open set is open for points $z_0$ where $f'(z_0) \neq 0$. \\

It remains to show that the open mapping theorem holds for points $z_0$ where $f'(z_0) = 0$. Consider $g(z) = f(z) - f(z_0)$. Clearly, $z_0$ is a zero of $g(z)$, as \[g(z_0) = f(z_0) - f(z_0) = 0.\] Furthermore, we know that $z_0$ is a zero of finite order, since $f(z)$ is not constant and thus is not equal to $f(z_0)$. Note also that $g'(z_0) = f'(z_0) - f'(z_0) = 0.$ Thus, we can conclude that $z_0$ is a zero of $g(z)$ of order $N \geq 2$.\\

We will now use our result from Exercise 9 (V.7.9). Note that since $g(z) = f(z) - f(z_0)$ is analytic (as $f(z)$ is analytic and $-f(z_0)$ is simply a translation by a constant) with zero of finite order $N \geq 2$ at $z_0$, we know from Exercise 9 that 
\[ g(z) = h(z)^N \]

for some $h(z)$ analytic near $z_0$ satisfying $h'(z_0) \neq 0$. Note that since $h'(z_0) \neq 0$, we know that locally around $z_0$, $h^{-1}(z)$ exists (from our previous case). By construction, then, $g^{-1}(z) = h^{-1}(\sqrt[m]{z})$ exists. Thus, since $g^{-1}(z)$ exists locally, we know that $g(z)$ is an open mapping. To conclude, note that
\[ f(z) = g(z) + f(z_0)\]

by construction. Since $g(z)$ is an open mapping and $f(z)$ is a translation of $g(z)$ by the constant $f(z_0),$ $f(z)$ is also an open mapping for points $z_0$ where $f'(z_0) = 0$. \\

Since the image under $f(z)$ of an open set $S$ is open for points $z_0 \in S$ where either $f'(z_0) \neq 0$ or $f'(z_0) = 0$, we conclude that the image under $f(z)$ of any open set is open, as desired.\end{solution}


\newpage

\underline{Section V.8 (Analytic Continuation), Problem 3} \\

\textbf{Show that each branch of $\sqrt{z}$ can be continued analytically along any path $\gamma$ in $\mathbb{C} \setminus \{0\},$ and show that the radius of convergence
of the power series $f_t(z)$ representing the continuation is $|\gamma(t)|.$ Show that $\sqrt{z}$ cannot be continued analytically along any path containing $0$.}

\begin{solution}
We will first determine the derivatives of $f(z) = \sqrt{z}$. Note that $f'(z) = \frac{1}{2\sqrt{z}}, f''(z) = \frac{1}{2} \cdot \left(-\frac{1}{2}\right)z^{-3/2}$, and in general, for $n \geq 1$,
\[f^{(n)}(z) = \frac{(-1)^{n+1} (2n-3)!!}{2^n} z^{\frac{-(2n-1)}{2}}.\]

Thus, the analytic continuation of $f(z) = \sqrt{z}$ along some path $\gamma$ in $\mathbb{C} \setminus \{0\}$ can be expressed by the following power series:
\begin{align*} f_t(\gamma(t)) &= \Sum \frac{f^{(n)}(\gamma(t))}{n!} (z-\gamma(t))^n  \\
&= f(\gamma(t)) + \sum\limits_{n=1}^{\infty} \frac{f^{(n)}(\gamma(t))}{n!} (z-\gamma(t))^n \\
&= f(\gamma(t)) + \sum\limits_{n=1}^{\infty} \left[(-1)^{n+1} \frac{(2n-3)!!}{2^n \cdot n!} (\gamma(t))^{-\frac{(2n-1)}{2}}\right] (z-\gamma(t))^n.\end{align*}

Since the analytic continuation of $f(z)$ is of this form and must be unique if it exists, we know that $\sqrt{z}$ cannot be analytically continued along any path through $0$, since the derivatives of $f(z) = \sqrt{z}$ are not defined at $z=0$. 
For any path $\gamma$ in $\mathbb{C} \setminus 0$, though, the analytic continuation is defined as above, and so each branch of $\sqrt{z}$ can be continued analytically along any path $\gamma$ in $\mathbb{C} \setminus \{0\}.$\\

\noindent\rule{\textwidth}{1pt} \\

We will use the ratio test to determine the radius of convergence of the power series $f_t(z)$. Note that
\begin{align*} \lim\limits_{k \rightarrow \infty} \left| \frac{a_k}{a_{k+1}}\right| &=  \lim\limits_{k \rightarrow \infty} \left| \frac{ (-1)^{k+1} \frac{(2k-3)!!}{2^k \cdot k!} (\gamma(t))^{-\frac{(2k-1)}{2}}}{(-1)^{k+2} \frac{(2k-1)!!}{2^{k+1} \cdot (k+1)!} (\gamma(t))^{-\frac{(2k+1)}{2}}}\right| \\
&= \lim\limits_{k \rightarrow \infty} \left| \frac{ (-1)^{k+1} \frac{(2k-3)!!}{2^k \cdot k!} (\gamma(t))^{-\frac{(2k-1)}{2}}}{(-1)^{k+2} \frac{(2k-1)(2k-3)!!}{2 \cdot 2^{k} \cdot (k+1)(k)!} (\gamma(t))^{-\frac{(2k+1)}{2}}}\right|.\end{align*}

Simplifying, we find that
\begin{align*} \lim\limits_{k \rightarrow \infty} \left| \frac{a_k}{a_{k+1}}\right| &=  \lim\limits_{k \rightarrow \infty} \left|-1 \cdot \frac{2(k+1)}{(2k-1)} \gamma(t) \right| \\
&= \lim\limits_{k \rightarrow \infty} \left|\frac{2k+2}{2k-1}\right| \left|\gamma(t) \right| \\
&= |\gamma(t)|.\end{align*}

Thus, by the ratio test, we find that the radius of convergence of the power series $f_t(z)$ representing the continuation is $|\gamma(t)|$, as desired. \\

Finally, note that if the path $\gamma(t)$ contains $0$, then the radius of convergence at $0$ in that path is $0$; thus, $\sqrt{z}$ cannot be continued analytically along any path containing $0$, as desired.\end{solution}

\end{document}
