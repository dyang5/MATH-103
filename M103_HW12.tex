\documentclass[11pt]{article}
\usepackage{graphicx}
\usepackage{amsthm}
\usepackage{amsmath}
\usepackage{amssymb}
\usepackage[shortlabels]{enumitem}
\usepackage[margin=1in]{geometry}

\newcommand{\C}{\mathbb{C}}
\newcommand{\D}{\mathbb{D}}

\newcommand{\Sum}{\sum\limits_{n=0}^{\infty}}
\newenvironment{solution}
  {\renewcommand\qedsymbol{$\blacksquare$}\begin{proof}[Solution]}
  {\end{proof}}

\setlength\parindent{0pt}

\newtheorem*{observation}{Observation}
\newtheorem*{theorem}{Theorem}
\newtheorem*{claim}{Claim}
\newtheorem*{corollary}{Corollary}

\theoremstyle{definition}
\newtheorem*{definition}{Definition}

\begin{document}

	\hrule
	\begin{center}
        \textbf{MATH103: Complex Analysis}\hfill \textbf{Fall 2023}\newline


		{\Large Homework 12}

		David Yang
	\end{center}

\hrule

\vspace{1em}


\textit{Chapter IX (The Schwarz Problems and Hyperbolic Geometry) Problems.} \\

\underline{Section IX.2 (Conformal Self-Maps of the Unit Disk), Problem 8}\\

\textbf{Show that every conformal self-map of the Riemann sphere $\C^*$ is given by a fractional linear transformation.}

\begin{solution}
Let $f(z)$ be a conformal self-map of the Riemann sphere $\C^*$. We know that $f(\infty) = \infty$ or $f(\infty) \neq \infty$, and we will consider these cases separately. \\

In the former case, $f(\infty) = \infty$. Consequently, since $f$ is a conformal self-map of $\C^*$ with a fixed point at $\infty$, $f$ must also be a conformal self-map of $\C$. 
By Exercise IX.2.7, we know that \[f(z) = az+b,\] with $a \neq 0$. Since $f(z) = az+b$ is a fractional linear transformation with $c=0$ and $d=1$, any conformal self-map $f(z)$ satisfying $f(\infty) = \infty$ is given by a fractional linear transformation. \\

In the latter case, $f(\infty) = c$ for some $c \neq \infty$. Consider the fractional linear transformation \[ g(z) = \frac{1}{z-c}.\]
Since $g(z)$ is a fractional linear transformation, it is a conformal self-map of $\C^*$. \\

Consider the function $g \circ f$. Since $g$ and $f$ are themselves conformal self-maps of $\C^*$ and the composition of conformal self-maps of $\C^*$ is also a conformal self-map of $\C^*$, we know $g \circ f$ is a conformal self-map of $\C^*$. 
Furthermore, note that \[(g \circ f)(\infty) = g(f(\infty)) = g(c) = \infty\]
so the function $g \circ f$ is a conformal self-map of $C^*$ with a fixed point at $\infty$. By our work above, we know that $g \circ f$ must be a fractional linear transformation of the form $az+b$. \\

Since $(g \circ f)(z) = \frac{1}{f(z) - c},$ we have that
\[ (g \circ f)(z) = \frac{1}{f(z) - c} = az+b\] 

with $a \neq 0$. Solving for $f(z)$ by taking the reciprocal of both sides and simplifying, we get that
\begin{align*} f(z) - c &= \frac{1}{az+b} \\
\Rightarrow f(z) &= c + \frac{1}{az+b} \\
\Rightarrow f(z) &= \frac{c(az+b)+1}{az+b}. \end{align*}

Simplifying, we get that
\[ f(z) = \frac{(ac)z + (bc+1)}{az+b}.\]

Note that $b(ac) - a(bc+1) = - a \neq 0$, and so $f(z)$ is a fractional linear transformation. \\

In both cases, we find that a conformal self-map $f(z)$ of $\C^*$ is a fractional linear transformation. Thus, every conformal self-map of the Riemann sphere $\C^*$ is given by a fractional linear transformation, as desired. \end{solution}    

\newpage

\underline{Section IX.2 (Conformal Self-Maps of the Unit Disk), Problem 13}\\

\textbf{Suppose $f(z)$ is an analytic function from the open unit disk $\D$ to itself that is not the identity map $z$.
Show that $f(z)$ has at most one fixed point in $\D$. \textit{Hint}. Make a change of variable with a conformal self-map of $\D$ to place the fixed point at $0$.}

\begin{solution}
We will prove the contrapositive; namely, that if $f(z)$ has at least two distinct fixed points, which we can denote $z_0$ and $z_1$ in $\D$, then $f(z)$ is the identity map. \\


Let $g(z)$ be the conformal self-map of $\D$ mapping $z_0$ to $0$ and $z_1$ to some nonzero value $c$: \[ g(z) = \frac{z-z_0}{1-\bar{z_0}z}.\]

Consider $h(z) = (g \circ f \circ g^{-1})(z).$ By definition, since $g$ and $g^{-1}$ are both conformal self-maps, they are analytic maps from $\D$ to $\D$. Similarly, $f$ is analytic from $\D$ to $\D$. 
Thus, $h$, the composition of these functions, is also an analytic function from $\D$ to $\D$. \\

Furthermore, note that
\[ h(0) = g(f(g^{-1}(0))) = g(f(z_0)) = g(z_0) = 0\]

by construction, as $g(z_0) = 0$, $g^{-1}(0) = z_0$, and $f(z_0) = z_0$ as $z_0$ is a fixed point of $f$ by assumption. \\

Similarly, consider the image of the nonzero value $c = g(z_1)$ under $h$: since by construction $g^{-1}(c) = z_1, g(z_1) = c$, and $f(z_1) = z_1$ as $z_1$ is a fixed point of $f$, we have
\[ h(c) = g(f(g^{-1}(c))) = g(f(z_1)) = g(z_1) = c,\]

giving us a nonzero fixed point for $h$. \\

By Schwarz Lemma, since $h$ is an analytic function from $\D$ to itself, $|h(z)| \leq 1$ (as $h$ is bounded by the unit disk) for $|z| < 1$ , and $h(0) = 0$, we know that
\[ |h(z)| \leq |z|.\] 

Since equality holds at $z = c \neq 0$, we know that by Schwarz Lemma, \[h(z) = \lambda z\] for some $\lambda$ of unit modulus. Even more, we must have that $\lambda =1$, since $h(c) = c$. Thus, $h(z) = z$. \\
Consequently, we know that
\[ h(z) = (g \circ f \circ g^{-1})(z) = z \]

for $z \in \D$. This tells us that
\[ (f \circ g^{-1})(z) = g^{-1}(z),\]

Treating $g^{-1}(z)$ as $z_1$ for some $z_1 \in D$, we must have that $f(z_1) = z_1$ for all $z_1 \in \D$. Consequently, $f$ is the identity function. \\

Thus, by proving the contrapositive statement, we know that $f(z)$ must have at most one fixed point in $\D$, as desired.\end{solution}

\end{document}