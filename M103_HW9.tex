\documentclass[11pt]{article}
\usepackage{graphicx}
\usepackage{amsthm}
\usepackage{amsmath}
\usepackage{amssymb}
\usepackage[shortlabels]{enumitem}
\usepackage[margin=1in]{geometry}

\newcommand{\C}{\mathbb{C}}
\newcommand{\Sum}{\sum\limits_{n=0}^{\infty}}
\newenvironment{solution}
  {\renewcommand\qedsymbol{$\blacksquare$}\begin{proof}[Solution]}
  {\end{proof}}

\setlength\parindent{0pt}

\newtheorem*{observation}{Observation}
\newtheorem*{theorem}{Theorem}
\newtheorem*{claim}{Claim}
\newtheorem*{corollary}{Corollary}

\theoremstyle{definition}
\newtheorem*{definition}{Definition}

\begin{document}

	\hrule
	\begin{center}
        \textbf{MATH103: Complex Analysis}\hfill \textbf{Fall 2023}\newline


		{\Large Homework 9}

		David Yang
	\end{center}

\hrule

\vspace{1em}


\textit{Chapter VII (The Residue Calculus) Problems.} \\

\underline{Section VII.8 (Exterior Domains), Problem 4}\\

\textbf{Show that}
\[ \int_{-1}^1 \frac{\sqrt{1-x^2}}{1+x^2} \, dx = \pi(\sqrt{2} - 1).\]

\begin{solution}
Consider the function $\frac{\sqrt{1-z^2}}{1+z^2}$ which has two analytic branches on the slit plane $\mathbb{C} \setminus [-1, 1]$, and let $f(z)$ be the branch of the function that is positive on the top edge of the slit $[0, 1]$. \\

We define the dogbone contour $\Gamma_\varepsilon$, consisting of segments from the top edge of the slit from $ -1 + \varepsilon$ to $1-\varepsilon$, a circle $\gamma_\varepsilon$ centered at $1$ of radius $\varepsilon$,
the bottom edge of the slit from $1-\varepsilon$ to $-1+\varepsilon$, and a circle $C_{\varepsilon}$ of radius $\varepsilon$ centered at $-1$. \\

Note that the phase factor of $f(z)$ at the branch points is $-1$, so $f(z)$ is negative on the bottom edge of the slit. Consequently, since the values of $f(z)$ on the bottom edge of the slit are minus those on the top edge, while the direction is reversed,
the integral along the bottom edge is equal to the integral along the top edge. Furthermore, by construction, we know that the integral along the top edge is simply $\int_{-1}^1 \frac{\sqrt{1-x^2}}{1+x^2} \, dx.$ \\

Thus, since the integral along the dogbone contour is the sum of the integrals along its path, we have that
\[ \int_{\Gamma_\varepsilon} \frac{\sqrt{1-z^2}}{1+z^2} \, dz = 2\int_{-1}^1 \, \frac{\sqrt{1-x^2}}{1+x^2} dx + \int_{C_\varepsilon} \frac{\sqrt{1-z^2}}{1+z^2} \, dz + \int_{\gamma_{\varepsilon}} \frac{\sqrt{1-z^2}}{1+z^2} \, dz. \, \, (\ast)\]

We can also bound the integrals over the circles $\gamma_{\varepsilon}$ and $C_{\varepsilon}$ using the ML-estimate:
\[ \left| \int_{\gamma_{\varepsilon}} \frac{\sqrt{1-z^2}}{1+z^2} \, dz \right| \leq 2\pi\varepsilon \cdot \left| \frac{\sqrt{1 - (1 - \varepsilon)^2}}{1 + (1-\varepsilon)^2 }\right| \]
\[ \left| \int_{C_{\varepsilon}} \frac{\sqrt{1-z^2}}{1+z^2} \, dz \right| \leq 2\pi\varepsilon \cdot \left| \frac{\sqrt{1 - (-1 + \varepsilon)^2}}{1 + (-1+\varepsilon)^2 }\right| \]

Note that when $\varepsilon \rightarrow 0$, the denominators of the fractions approach a constant value, whereas the numerators approach $0$. Thus, each of these estimates approach $0$ as $\varepsilon \rightarrow 0$. \\

Consequently, we can rewrite equation $(\ast)$ as
\[ \int_{\Gamma_\varepsilon} \frac{\sqrt{1-z^2}}{1+z^2} \, dz = 2\int_{-1}^1 \, \frac{\sqrt{1-x^2}}{1+x^2} dx. \, \, (\Diamond) \]

We can now evaluate the left-hand side of this equation. The function $\frac{\sqrt{1-z^2}}{1+z^2}$ has simple poles at $\pm i$ and at the point at $\infty$. Consequently, we know by the Residue Theorem for Exterior Domains that
\[ \int_{\Gamma_\varepsilon} \frac{\sqrt{1-z^2}}{1+z^2} \, dz = 2\pi i \left( \mathrm{Res}\left[\frac{\sqrt{1-z^2}}{1+z^2}, i\right]
+ \mathrm{Res}\left[ \frac{\sqrt{1-z^2}}{1+z^2}, -i \right] + \mathrm{Res}\left[ \frac{\sqrt{1-z^2}}{1+z^2}, \infty \right]  \right).\]

Note that by Rule 3, since $i$ is a simple zero of $1+z^2$, 
\[ \mathrm{Res}\left[\frac{\sqrt{1-z^2}}{1+z^2}, i\right] = \frac{\sqrt{1-(i)^2}}{2(i)} = \frac{\sqrt{2}}{2i}.\]

Similarly, since $-i$ is a simple zero of $1+z^2$, 
\[ \mathrm{Res}\left[ \frac{\sqrt{1-z^2}}{1+z^2}, -i \right] = \frac{\sqrt{1-(i)^2}}{2(-i)} = -\frac{\sqrt{2}}{2i}.\]

However,  this does not account for the phase change (with phase factor $-1$) which occurs as we traverse the dogbone contour. 
Accounting for this phase changes tells us that
\[ \mathrm{Res}\left[ \frac{\sqrt{1-z^2}}{1+z^2}, -i \right] = \frac{\sqrt{2}}{2i}.\]

Finally, by Exercise 7.8.13, we know that to evaluate the residue of $f(z) = \frac{\sqrt{1-z^2}}{1+z^2}$ at $\infty$, we can equivalently evaluate the residue of $f(-1/w)/w^2$ at $w=0$. Consequently, we have
\begin{align*} \mathrm{Res}\left[ \frac{\sqrt{1-z^2}}{1+z^2}, \infty \right] &= \mathrm{Res}\left[ -\frac{1}{w^2}\frac{\sqrt{1-\left(\frac{1}{w}\right)^2}}{1+\left(\frac{1}{w}\right)^2}, 0 \right] \\
&= \mathrm{Res}\left[-\frac{\sqrt{1-\left(\frac{1}{w}\right)^2}}{w^2+1}, 0 \right] \\
&=  \mathrm{Res}\left[ -\frac{\sqrt{w^2 - 1}}{w(w^2+1)}, 0 \right]
\end{align*}

To evaluate this residue, we will make an observation about the branch of $\sqrt{1-w^2}$. Note that
\[ \sqrt{1-w^2} = i\sqrt{w-1}\sqrt{w+1} \text{ or } -i\sqrt{w-1}\sqrt{w+1}.\]

However, since $\sqrt{1-w^2}$ should be positive at $w=0$, we want to work with the branch 
\[\sqrt{1-w^2}= -i\sqrt{w-1}\sqrt{w+1}\] to maintain this fact. \\

By Rule 3, since $0$ is a simple zero of $w(w^2+1)$ (as the derivative $3w^2 + 1$ is nonzero at $w=0$), we find that
\[ \mathrm{Res}\left[ -\frac{\sqrt{w^2 - 1}}{w(w^2+1)}, 0 \right] = -\frac{\sqrt{w^2 - 1}}{3w^2 + 1} \bigg\rvert_{z=0}.\]

By our above reasoning, we know that $\sqrt{w^2 - 1} = i\sqrt{1-w^2} = i(-i)\sqrt{w-1}\sqrt{w+1}$, and so evaluating this at $w=0$ gives $i(-i)(i) = i.$ 
Thus, the residue of $\frac{\sqrt{1-z^2}}{1+z^2}$ at $\infty$ is $i$. \\

Plugging these residues back into our previous equation, we have that
\begin{align*} \int_{\Gamma_\varepsilon} \frac{\sqrt{1-z^2}}{1+z^2} \, dz &= 2\pi i \left( \mathrm{Res}\left[\frac{\sqrt{1-z^2}}{1+z^2}, i\right]
+ \mathrm{Res}\left[ \frac{\sqrt{1-z^2}}{1+z^2}, -i \right] + \mathrm{Res}\left[ \frac{\sqrt{1-z^2}}{1+z^2}, \infty \right]  \right) \\
&= 2\pi i \left( \frac{\sqrt{2}}{2i} + \frac{\sqrt{2}}{2i} + i \right) \\
&= 2\pi(\sqrt{2} - 1).
\end{align*}

Recall that from equation $(\Diamond)$,
\[ 2\int_{-1}^1 \, \frac{\sqrt{1-x^2}}{1+x^2} dx = \int_{\Gamma_\varepsilon} \frac{\sqrt{1-z^2}}{1+z^2} \, dz. \]

Plugging in our above calculation for $\int_{\Gamma_\varepsilon} \frac{\sqrt{1-z^2}}{1+z^2} \, dz$, we find that
\[  2\int_{-1}^1 \, \frac{\sqrt{1-x^2}}{1+x^2} dx = 2\pi(\sqrt{2} - 1),\]

and dividing both sides by two gives us the desired result:
\[ \boxed{\int_{-1}^1 \frac{\sqrt{1-x^2}}{1+x^2} \, dx = \pi(\sqrt{2} - 1)}.\]
\end{solution}

\newpage

\underline{Section VII.8 (Exterior Domains), Problem 4}\\

\textbf{Show that the analytic differential $f(z) \, dz$ transforms under the change of variable $w=1/z$ to $-f(1/w) \, dw/w^2.$ Show that the residue of $f(z)$ at $z = \infty$
coincides with that of $-f(1/w)/w^2$ at $w=0$.} \\

\begin{solution} Consider $f(z) \, dz$, and apply the change of variable $w=1/z,$ or $z=1/w.$ Clearly, we also have that $dz = -1/w^2 \, dw.$ Thus, substituting, we find that
\[ f(z) \, dz = f\left( 1/w \right) \left(  -1/w^2 \right) dw = -f\left( 1/w\right) \, dw/w^2\]

as desired. 

\noindent\rule{\textwidth}{1pt} \\

Since $f(z)$ is analytic at $\infty$, it is analytic for some exterior domain. Consequently, we can write $f(z)$ as its Laurent series expansion
\[ f(z) = \sum\limits_{n = -\infty}^{\infty} a_n z^n. \]

By definition, the residue of $f(z)$ at $z = \infty$ is $-a_{-1}.$ \\

Since $f(z)$ is analytic at $\infty$, we know that $f(1/w)$ is analytic at $0$. Furthermore, the Laurent expansion of $f(1/w)$ at $w=0$ is
\[ f(1/w) = \sum\limits_{n = -\infty}^{\infty} a_n \left(1/w\right)^n = \sum\limits_{n = -\infty}^{\infty} a_n w^{-n}. \]

It follows that the Laurent expansion of $\frac{-f\left( 1/w\right)}{w^2}$ is
\[ \frac{-f\left( 1/w\right)}{w^2} = -\frac{1}{w^2}\sum\limits_{n = -\infty}^{\infty} a_n w^{-n} = \sum\limits_{n=-\infty}^{\infty} -a_n w^{-n-2}.\]

By definition, the residue of $f(1/w)$ at $w=0$ is the coefficient of the $\frac{1}{w}$ term (corresponding to $n= -1$), which is also $-a_{-1}.$ \\

Thus, we see that the residue of $f(z)$ at $z = \infty$ coincides with that of $-f(1/w)/w^2$ at $w=0$, as desired. \end{solution}
\end{document}
