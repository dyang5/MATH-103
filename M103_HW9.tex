\documentclass[11pt]{article}
\usepackage{graphicx}
\usepackage{amsthm}
\usepackage{amsmath}
\usepackage{amssymb}
\usepackage[shortlabels]{enumitem}
\usepackage[margin=1in]{geometry}

\newcommand{\C}{\mathbb{C}}
\newcommand{\Sum}{\sum\limits_{n=0}^{\infty}}
\newenvironment{solution}
  {\renewcommand\qedsymbol{$\blacksquare$}\begin{proof}[Solution]}
  {\end{proof}}

\setlength\parindent{0pt}

\newtheorem*{observation}{Observation}
\newtheorem*{theorem}{Theorem}
\newtheorem*{claim}{Claim}
\newtheorem*{corollary}{Corollary}

\theoremstyle{definition}
\newtheorem*{definition}{Definition}

\begin{document}

	\hrule
	\begin{center}
        \textbf{MATH103: Complex Analysis}\hfill \textbf{Fall 2023}\newline


		{\Large Homework 7}

		David Yang
	\end{center}

\hrule

\vspace{1em}


\textit{Chapter VII (The Residue Calculus) Problems.} \\

\underline{Section VII.8 (Exterior Domains), Problem 4}\\

\textbf{Show that}
\[ \int_{-1}^1 \frac{\sqrt{1-x^2}}{1+x^2} \, dx = \pi(\sqrt{2} - 1).\]

\begin{solution}
Consider the functrion $\frac{1-z^2}{1+z^2}$ which has two analytic branches on the slit plane $\mathbb{C} \setminus [-1, 1]$, and let $f(z)$ be the branch of the function that is positive on the top edge of the slit $[0, 1]$. \\

We define the dogbone contour $\Gamma_\varepsilon$, consisting of segments from the top edge of the slit from $ -1 + \varepsilon$ to $1-\varepsilon$, a circle $\gamma_\varepsilon$ centered at $1$ of radius $\varepsilon$,
the bottom edge of the slit from $1-\varepsilon$ to $-1+\varepsilon$, and a circle $C_{\varepsilon}$ of radius $\varepsilon$ centered at $-1$. \\

Note that the phase factor of $f(z)$ at the branch points is $-1$, so $f(z)$ is negative on the bottom edge of the slit. Consequently, since teh values of $f(z)$ on the bottom edge of the slit are minus those on the top edge, while the direction is reversed,
the integral along the bottom edge is equal to the integral along the top edge.
\end{solution}

\newpage

\underline{Section VII.8 (Exterior Domains), Problem 4}\\

\textbf{Show that the analytic differential $f(z) \, dz$ transforms under the change of variable $w=1/z$ to $-f(1/w) \, dw/w^2.$ Show that the residue of $f(z)$ at $z = \infty$
coincides with that of $-f(1/w)/w^2$ at $w=0$.} \\

\begin{solution} Consider $f(z) \, dz$, and apply the change of variable $w=1/z,$ or $z=1/w.$ Clearly, we also have that $dz = -1/w^2 \, dw.$ Thus, substituting, we find that
\[ f(z) \, dz = f\left( 1/w \right) \left(  -1/w^2 \right) dw = -f\left( 1/w\right) \, dw/w^2\]

as desired. 

\noindent\rule{\textwidth}{1pt} \\

Since $f(z)$ is analytic at $\infty$, it is analytic for some exterior domain. Consequently, we can write $f(z)$ as its Laurent series expansion
\[ f(z) = \sum\limits_{n = -\infty}^{\infty} a_n z^n. \]

By definition, the residue of $f(z)$ at $z = \infty$ is $-a_{-1}.$ \\

Since $f(z)$ is analytic at $\infty$, we know that $f(1/w)$ is analytic at $0$. Furthermore, the Laurent expansion of $f(1/w)$ at $w=0$ is
\[ f(1/w) = \sum\limits_{n = -\infty}^{\infty} a_n \left(1/w\right)^n = \sum\limits_{n = -\infty}^{\infty} a_n w^{-n}. \]

It follows that the Laurent expansion of $\frac{-f\left( 1/w\right)}{w^2}$ is
\[ \frac{-f\left( 1/w\right)}{w^2} = -\frac{1}{w^2}\sum\limits_{n = -\infty}^{\infty} a_n w^{-n} = \sum\limits_{n=-\infty}^{\infty} -a_n w^{-n-2}.\]

By definition, the residue of $f(1/w)$ at $w=0$ is the coefficient of the $\frac{1}{w}$ term (corresponding to $n= -1$), which is also $-a_{-1}.$ \\

Thus, we see that the residue of $f(z)$ at $z = \infty$ coincides with that of $-f(1/w)/w^2$ at $w=0$, as desired. \end{solution}
\end{document}
