\documentclass[11pt]{article}
\usepackage{graphicx}
\usepackage{amsthm}
\usepackage{amsmath}
\usepackage{amssymb}
\usepackage[shortlabels]{enumitem}
\usepackage[margin=1in]{geometry}

\newcommand{\C}{\mathbb{C}}
\newcommand{\Sum}{\sum\limits_{n=0}^{\infty}}
\newenvironment{solution}
  {\renewcommand\qedsymbol{$\blacksquare$}\begin{proof}[Solution]}
  {\end{proof}}

\setlength\parindent{0pt}

\newtheorem*{observation}{Observation}
\newtheorem*{theorem}{Theorem}
\newtheorem*{claim}{Claim}
\newtheorem*{corollary}{Corollary}

\theoremstyle{definition}
\newtheorem*{definition}{Definition}

\begin{document}

	\hrule
	\begin{center}
        \textbf{MATH103: Complex Analysis}\hfill \textbf{Fall 2023}\newline


		{\Large Homework 11}

		David Yang
	\end{center}

\hrule

\vspace{1em}


\textit{Chapter VIII (The Logarithmic Integral) Problems.} \\

\underline{Section VIII.8 (Simply Connected Domains), Problem 4}\\

\textbf{Show that a domain $D$ in the complex plane is simply connected if and only if any analytic function $f(z)$ on $D$ that does not vanish at any point of $D$ has an analytic logarithm on $D$. \textit{Hint.} If $f(z) \neq 0$ on $D$, consider the function}

\[ G(z) = \int_{z_0}^z \frac{f'(w)}{f(w)} \, dw. \]

\begin{solution}
We will begin by proving the forward implication. Suppose that $D$ is a simply connected domain in the complex plane. By property (ii) of the Theorem on page 254, we know that every closed differential on $D$ is exact. Consider an analytic function $f(z)$ on $D$ that does not vanish at any point of $D$, i.e. $f(z) \neq 0$ on $D$, and the function
\[ G(z) = \int_{z_0}^z \frac{f'(w)}{f(w)} \, dw.\]

Since $f(w)$ is analytic on $D$, so is $f'(w)$. Furthermore, $f$ does not vanish at any point of $D$, so $f(w) \neq 0$. Thus, $\frac{f'(w)}{f(w)}$ is analytic for all $w \in D$ and consequently, $G(z)$ is analytic on $D$. \\

Furthermore, we know that
\begin{align*} G(z) &= \int_{z_0}^z \frac{f'(w)}{f(w)} \, dw \\
&= \int_{z_0}^z d\log(f(w)) \end{align*}

since every closed differential on $D$ is exact. Thus, we have an analytic logarithm of $f(z)$ on $D$, as desired. \\

For the reverse implication, we assume that any analytic function $f$ on $D$ that does not vanish at any point on $D$ has an analytic logarithm on $D$. Consider the function $f(w) = w - z_0$, for some $z_0 \in \mathbb{C} \setminus D$, which does not vanish at any point on $D$. \\

Consider, for any closed path $\gamma$ in $D$, \begin{align*}
G(z) &= \frac{1}{2\pi i}\int_{\gamma} d\log(f(w)) \\
&= \frac{1}{2\pi i}\int_{\gamma} \frac{f'(w)}{f(w)} \, dw.
\end{align*}

Substituting $f'(w) = 1$ and $f(w) = w - z_0$ into the integrand, we find that
\[ G(z) = \frac{1}{2\pi i} \int_{\gamma} \frac{1}{w-z_0} \, dw = W(\gamma, z_0) = 0.\]

where the equivalence to the winding number follows by definition. Thus, by property (iv) of the Theorem on page 254, we find that $D$ is simply connected, as desired.
\end{solution}

\newpage

\underline{Extra Problem}\\

\textbf{Assume that $f$ is analytic and $|f(z)| < 1$ on the set $\{ |z| \leq 1\}.$ Use Rouche's Theorem to show that $f$ has a fixed point. }

\begin{solution}
Consider the function $g(z) = -z$ and the disk $D = \{ |z| < 1\}.$ Note that
\[ |f(z)| < |g(z)| = |z|\]

for $z \in \partial D$, since $|z| = 1$ on $\partial D.$ \\

By Rouche's Theorem, we know that $g(z) = -z$ and $f(z) + g(z) = f(z) - z$ have the same number of zeros in $D$. Since $g(z)= -z$ has one zero at $z=0$, we know that $f(z) - z$ has a zero in $D$, meaning $f(z_0) = z_0$ for some $z_0 \in D$. Equivalently, $z_0$ is a fixed point of $f(z)$, as desired..\end{solution}

\end{document}