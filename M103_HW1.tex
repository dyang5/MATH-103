\documentclass[11pt]{article}
\usepackage{graphicx}
\usepackage{amsthm}
\usepackage{amsmath}
\usepackage{amssymb}
\usepackage[shortlabels]{enumitem}
\usepackage[margin=1in]{geometry}

\newcommand{\C}{\mathbb{C}}

\newenvironment{solution}
  {\renewcommand\qedsymbol{$\blacksquare$}\begin{proof}[Solution]}
  {\end{proof}}

\setlength\parindent{0pt}

\newtheorem*{observation}{Observation}
\newtheorem*{theorem}{Theorem}
\newtheorem*{claim}{Claim}
\newtheorem*{corollary}{Corollary}

\theoremstyle{definition}
\newtheorem*{definition}{Definition}

\begin{document}

	\hrule
	\begin{center}
        \textbf{MATH103: Complex Analysis}\hfill \textbf{Fall 2023}\newline

		{\Large Homework 1}

		David Yang
	\end{center}

\hrule

\vspace{1em}

\textit{Chapter I (The Complex Plane and Elementary Functions) Problems.} \\

\underline{Section I.3 (Stereographic Projection), I.3.4} \\

\textbf{Show that a rotation of the sphere of $180^{\circ}$ about the $X$-axis corresponds under stereographic projection to the inversion $z \mapsto \frac{1}{z}$ of $\mathbb{C}$.} \\

\begin{solution}
    Let $P = (X, Y, Z)$ be a point on the unit sphere. After a a $180^{\circ}$ rotation of the point $P$ on the unit sphere about the $X$-axis, $P$ is sent to the point $P' = (X, -Y, -Z)$. \\
    
    Consider the result of $P$ and $P'$ under stereographic projection. By definition, stereographic projection sends $P$ to the point 
    \[\frac{X}{1-Z} + \frac{Y}{1-Z}i\] 
    
    and the point $P'$ to the point 
    \[ \frac{X}{1-(-Z)} + \frac{-Y}{1-(-Z)}i = \frac{X}{1+Z} - \frac{Y}{1+Z}i \]

    on the extended complex plane $\C^{*}$. \\

    We claim that $P'$ is the result of $P$ under the inversion $z \mapsto \frac{1}{z}$ of $\C$; note that
    \[ \left( \frac{X}{1-Z} + \frac{Y}{1-Z}i \right) \left( \frac{X}{1+Z} - \frac{Y}{1+Z}i \right)\]
    \[ = \frac{X^2}{(1-Z)(1+Z)} - \frac{XY}{(1-Z)(1+Z)} + \frac{XY}{(1-Z)(1+Z)} - \frac{Y^2}{(1-Z)(1+Z)} i^2. \]

    By using the identity $i^2 = -1$, canceling out terms, and simplifying, we find that this is

    \[ \frac{X^2}{(1-Z)(1+Z)} +\frac{Y^2}{(1-Z)(1+Z)} = \frac{X^2 + Y^2}{1-Z^2}.\]

    However, since $P = (X, Y, Z)$ is a point on the unit sphere, we know that $X^2 + Y^2 + Z^2 = 1$, so $1-Z^2 = X^2+Y^2$. Thus, we know that 

    \[ \left( \frac{X}{1-Z} + \frac{Y}{1-Z}i \right) \left( \frac{X}{1+Z} - \frac{Y}{1+Z}i \right) = \frac{X^2+Y^2}{1-Z^2} = 1.\]

    which tells us that the resulting points of $P$ and $P'$ under stereographic projection are complex inverses. \\
    
    Thus, a rotation of the sphere of $180^{\circ}$ about the $X$-axis corresponds under stereographic projection to the inversion $z \mapsto \frac{1}{z}$ of $\mathbb{C}$.
\end{solution}

\newpage

\underline{Section I.6 (The Logarithm Function), I.6.2} \\

\textbf{Sketch the image under the map $w = \mathrm{Log} \, z$ of each of the following figures:}

\begin{enumerate}[a)]
\item \textbf{the right half-plane $\mathrm{Re} \, z > 0$} \\

We know that \[\mathrm{Log} \, z = \log |z| + i \, \mathrm{Arg} \, z.\]

For points in the right half-plane, $-\frac{\pi}{2} < \mathrm{Arg} \, z < \frac{\pi}{2}$. Thus, the image in the $w$-plane under the logarithm function satisfies $\frac{-\pi}{2} < \mathrm{Im} w < \frac{\pi}{2}.$ \\

On the other hand, there is no restriction on $|z|$; all values of $|z| > 0$ are in the right half-plane, and so $\mathrm{Re} \, w \in (-\infty, \infty)$. 

\begin{center}
\includegraphics*[scale = 0.15]{I.6.2a.jpeg}
\end{center}

\item[c)] \textbf{the unit circle $|z| = 1$} \\

We know that \[\mathrm{Log} \, z = \log |z| + i \, \mathrm{Arg} \, z.\]

On the unit circle, $|z| = 1$, so 
\[ w = \mathrm{Log} \, z = \log (1) + i \, \mathrm{Arg} \, z = i \, \mathrm{Arg} \, z.\]

On the unit circle, $\mathrm{Arg} \, z$ takes on all values in $[-\pi, \pi]$, so the image is simply the line segment $\{i y : y \in [-\pi, \pi]\}.$ 


\begin{center}
  \includegraphics*[scale = 0.15]{I.6.2c.jpeg}
\end{center}


\item[e)] \textbf{the horizontal line $y=e$} \\

We know that \[\mathrm{Log} \, z = \log |z| + i \, \mathrm{Arg} \, z.\]

Since $y = e$, note that the minimum value for $|z| = e$, which occurs when $\mathrm{Re} \, z = 0.$ For $z = e$, \[w = \mathrm{Log} \, z = \log e + \mathrm{Arg} \, e = 1 + \frac{\pi}{2}i.\]

Consider the behavior of $w$ as $z$ moves along the right half of the line $y = e$ (i.e. $\mathrm{Re} \, z > 0$). As $z$ moves along this section, $|z|$ increases without bound and $\mathrm{Arg} \, z$ approaches $\frac{\pi}{2}$.
Furthermore, we know that $|z|$ (and $\log z$, which is in turn the real part of the image $w$) grows quicker the sooner we move along this section (when $\mathrm{Re} \, z$ is small, changes in $|z|$ are relatively large). \\

We can apply a similar line of reasoning to determine the behavior of the left half of the line, where $\mathrm{Re} \, z < 0$ and $\mathrm{Im} \, z = e$. In this case, $\mathrm{Arg} \, z$ approaches $0$, and so we arrive at the following image (the image has a reflective symmetry at $y = \frac{\pi}{2}$ in the $w$-plane): 

\begin{center}
  \includegraphics*[scale = 0.15]{I.6.2e.jpeg}
\end{center}

\end{enumerate}

\newpage

\underline{Section I.8 (Trigonometric and Hyperbolic Functions), I.8.5} \\

\textbf{Let $S$ denote the two slits along the imaginary axis in the complex plane, one running from $i$ to $+i\infty$, the other running from $-i$ to $-i\infty$.}

\begin{enumerate}[a)]
\item \textbf{Show that $\frac{1+iz}{1-iz}$ lies on the negative real axis $(-\infty, 0]$ if and only if $z \in S$.}

\begin{solution}

We will begin by proving the forward direction and showing that if $\frac{1+iz}{1-iz}$ lies on the negative real axis $(-\infty, 0]$ then $z \in S$. \\

Let $\frac{1+iz}{1-iz}$ lie on the negative real axis, meaning that \[ \frac{1+iz}{1-iz} = r \] for real $r \in (-\infty, 0].$ Multiplying both sides by $1-iz$, we get that
\[ 1+iz = r(1-iz).\]

Moving the imaginary terms to one side and the real terms to the other, we get that \[zi(1+r) = r-1\] and solving for $z$ gives us
\[ z = \frac{1-r}{1+r}i. \]

Note that when $r=0$, $z=i$, and as $r \rightarrow -1$ from $0$, $z$ runs from $i$ to \[\lim\limits_{r \rightarrow -1^{+}} \frac{1-r}{1+r}i = i\infty\]

which is the right slit along the imaginary axis in the complex plane. \\

On the other hand, as $r \rightarrow -\infty$, we know that $z$ approaches \[ \lim\limits_{r \rightarrow -\infty} \frac{1-r}{1+r}i = -i\] and as $z$ approaches $-1$ from the right, $z$ runs from $-i$ to
\[ \lim\limits_{r \rightarrow -1^{-}} \frac{1-r}{1+r}i = -i\infty\]

which is the left slit along the imaginary axis in the complex plane. \\

Thus, if $\frac{1+iz}{1-iz}$ lies on the negative real axis $(-\infty, 0]$ then $z \in S$. \\


To prove the reverse direction, we want to show that of $z \in S$, then $\frac{1+iz}{1-iz}$ lies on the negative real axis $(-\infty, 0].$ \\

By definition, if $z \in S$, then $z=ci$ for some $c \in \mathbb{R}$ satisfying $c \in (-\infty, -1) \cup [1, \infty).$ Plugging this value for $z$ into $\frac{1+iz}{1-iz}$, we get that
\[ \frac{1+iz}{1-iz} = \frac{1+i(ci)}{1-i(ci)} = \frac{1-c}{1+c}\]


Note that when $c=1$, this expression evaluates to $0$. On the other hand, the expression $\frac{1-c}{1+c}$ is positive if and only if both $1-c$ and $1+c$ are negative,
or if both $1-c$ and $1+c$ are positive. Note that the former case cannot occur as if $1-c < 0$, then $c>1$ which would make $1+c$ positive. Similarly, the latter case can only occur when $-1 < c < 1$, which violates
the condition that $c \in (-\infty, -1) \cup [1, \infty).$ \\

Thus, if $z \in S$, then $\frac{1+iz}{1-iz}$ must lie on the negative real axis $(-\infty, 0].$ \\

Since we have proved both directions of the if and only if, we know that $\frac{1+iz}{1-iz}$ lies on the negative real axis $(-\infty, 0]$ if and only if $z \in S$. \end{solution}
\item \textbf{Show that the principal branch 
\[ \mathrm{Tan}^{-1} z = \frac{1}{2i} \, \mathrm{Log} \left( \frac{1+iz}{1-iz} \right) \]
maps the slit plane $\C \setminus S$ one-to-one onto the vertical strip $\{ | \mathrm{Re} \, w | < \frac{\pi}{2}\}$.}

\begin{solution}
We will first show that this map is one-to-one. Let $a, b \in \C \setminus S$ and assume that $\mathrm{Tan}^{-1}(a) = \mathrm{Tan}^{-1}(b)$. We will show that $a = b$. If $\mathrm{Tan}^{-1}(a) = \mathrm{Tan}^{-1}(b)$, we know that

\[\frac{1}{2i} \, \mathrm{Log} \left( \frac{1+ia}{1-ia} \right) = \frac{1}{2i} \, \mathrm{Log} \left( \frac{1+ib}{1-ib} \right). \]

Multiplying both sides by $2i$ and applying the exponential function (which we can do since neither $a$ and $b$ are in $S$, so by part (a), $\frac{1+ia}{1-ia}$ and $\frac{1+ib}{1-ib}$ will not lie on the negative imaginary axis) to both sides, we get that
\[ \frac{1+ia}{1-ia} = \frac{1+ib}{1-ib}.\]

Cross multiplying, we get that \[ (1+ia)(1-ib) = (1+ib)(1-ia). \]

Expanding and simplifying, we get that
\[ (ab + 1) + i(a-b) = (ab + 1) + i(b-a). \]

Subtracting both sides by $ab+1$ and dividing by $i$, we get that $a-b = b-a$, meaning $a = b$.

Thus, this map is one-to-one. Note that the principal logarithm of $z \in \C \setminus S$ has imaginary part $i \mathrm{Arg} \, z.$ Since $-\pi < \mathrm{Arg} \, z < \pi$ (the latter inequality coming from the fact that $z \in \C \setminus S$), we know that 
\[ |\mathrm{Re} \, w| = |\mathrm{Re}(\mathrm{Tan}^{-1}(z))| = \left|\mathrm{Re}\left(\frac{1}{2i} \mathrm{Log} \left( \frac{1+iz}{1-iz} \right)\right)\right| < \frac{\pi}{2}. \]

Thus, we know that this map maps the slit plane one-to-one onto the vertical strip. 
\end{solution}


\end{enumerate}

\end{document}
