\documentclass[11pt]{article}
\usepackage{graphicx}
\usepackage{amsthm}
\usepackage{amsmath}
\usepackage{amssymb}
\usepackage[shortlabels]{enumitem}
\usepackage[margin=1in]{geometry}

\newcommand{\C}{\mathbb{C}}

\newenvironment{solution}
  {\renewcommand\qedsymbol{$\blacksquare$}\begin{proof}[Solution]}
  {\end{proof}}

\setlength\parindent{0pt}

\newtheorem*{observation}{Observation}
\newtheorem*{theorem}{Theorem}
\newtheorem*{claim}{Claim}
\newtheorem*{corollary}{Corollary}

\theoremstyle{definition}
\newtheorem*{definition}{Definition}

\begin{document}

	\hrule
	\begin{center}
        \textbf{MATH103: Complex Analysis}\hfill \textbf{Fall 2023}\newline

		{\Large Homework 4}

		David Yang
	\end{center}

\hrule

\vspace{1em}


\textit{Chapter III (Line Integrals and Harmonic Functions) Problems.} \\

\underline{Section III.3 (Harmonic Conjugates), Problem 3} \\

\textbf{Let $D = \{a < |z| < b\} \setminus (-b, -a)$, an annulus slit along the neagtive real axis. Show that any harmonic function on $D$ has a harmonic conjugate on $D$. \textit{Suggestion}. Fix $c$ between $a$ and $b$, and define $v(z)$ explicitly as a 
line integral along the path consisting of the straight line from $c$ to $|z|$ followed by the circular arc from $|z|$ to $z$. Or map the slit annulus to a rectangle by $w = \mathrm{Log} z.$}

\begin{solution}
\end{solution}

\underline{Section III.4 (The Mean Value Property), Problem 4} \\

\textbf{Formulate the mean value property for a function on a domain in $\mathbb{R}^3,$ and show that any harmonic function has the mean value property. \textit{Hint}. For $A \in \mathbb{R}^3$ amd $r> 0$, let $B_r$ be the ball of radius $r$
centered at $A$, with volume element $d\tau$, and let $\partial B_r$ be its boundary sphere, with area element $d\sigma$ and unit outward normal vector $\textbf{n}$. 
Apply the Gauss divergence theorem \[ \int\int_{\partial B_r} \textbf{F} \cdot \textbf{n} \, d\sigma = \int\int\int_{B_r} \nabla \cdot \textbf{F} \, d\tau \] to $\textbf{F} = \triangle u.$}

\begin{solution}
    
\end{solution}

\underline{Section III.5 (The Maximum Principle), Problem 3} \\

\textbf{Use the maximum principle to prove the fundamental theorem of algebra, that any polynomial $p(z)$ of degree $n \geq 1$ has a zero, by applying the maximum principle
to $1/p(z)$ on a disk of a large radius.}

\begin{solution}
    
\end{solution}
\end{document}
