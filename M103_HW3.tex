\documentclass[11pt]{article}
\usepackage{graphicx}
\usepackage{amsthm}
\usepackage{amsmath}
\usepackage{amssymb}
\usepackage[shortlabels]{enumitem}
\usepackage[margin=1in]{geometry}

\newcommand{\C}{\mathbb{C}}

\newenvironment{solution}
  {\renewcommand\qedsymbol{$\blacksquare$}\begin{proof}[Solution]}
  {\end{proof}}

\setlength\parindent{0pt}

\newtheorem*{observation}{Observation}
\newtheorem*{theorem}{Theorem}
\newtheorem*{claim}{Claim}
\newtheorem*{corollary}{Corollary}

\theoremstyle{definition}
\newtheorem*{definition}{Definition}

\begin{document}

	\hrule
	\begin{center}
        \textbf{MATH103: Complex Analysis}\hfill \textbf{Fall 2023}\newline

		{\Large Homework 3}

		David Yang
	\end{center}

\hrule

\vspace{1em}


\textit{Chapter II (Analytic Functions) Problems.} \\

\underline{Section II.6 (Conformal Mappings), II.6.6}

\begin{enumerate}[a)]
    \item \textbf{Determine where the function $f(z) = z + 1/z$ is conformal and where it is not conformal.}
    \begin{solution} We know that $f(z)$ is conformal at $z_0$ if it is analytic at $z_0$ where $f'(z_0) \neq 0.$ Since $f(0)$ is not defined, we know $f$ cannot be conformal at the origin. Similarly, we see that
    \[ f'(z) = 1 - \frac{1}{z^2} \]

    and so $f'(z) \neq 0$ when $z \neq \pm 1.$ Since $f(z) \neq 0$ at all other points, $f$ is also analytic at all other points. Thus, $f(z) = z + 1/z$ is not conformal for $z = 0, \pm 1$ and is conformal at all other points. \end{solution}

    \item \textbf{Show that for each $w$, there are at most two values $z$ for which $f(z) = w.$} 
    \begin{solution} Fix some complex $w$. The values $z$ for which $f(z) = w$ must by definition satisfy
    \[ z + \frac{1}{z} = w.\]

    Multiplying both sides by $z$ (which we can do as $z \neq 0$) and rearranging, we get that
    \[ z^2 - zw + 1 = 0. \]

    Since this is a quadratic with respect to $z$, there are at most two values $z$ that solve this equation, as desired. (In fact, by Vieta's Formulas, we know that the roots $z_1$ and $z_2$ of $f(z) = w$ must satisfy $z_1z_2 = 1$.) \end{solution}

    \item \textbf{Show that if $r > 1$, $f(z)$ maps the circle $\{|z| = r\}$ onto an ellipse, and that $f(z)$ maps the circle $\{|z| = 1/r\}$ onto the same ellipse. }
    
    \begin{solution} We will first show that if $r > 1$, $f(z)$ maps the circle $\{|z| = r\}$ onto an ellipse. Writing $z$ in polar form, we know that $z=re^{i\theta}.$ Thus, we know that
    \[f(z) = z + 1/z = re^{i\theta} + \frac{1}{r}e^{-i\theta}.\]

    Substituting $e^{i\theta} = \cos\theta + i\sin \theta$ and $e^{-i\theta} = \cos(-\theta) + i\sin(-\theta) = \cos \theta - i\sin\theta$, into this expression, we find that
    \begin{align*} f(z) &= re^{i\theta} + \frac{1}{r}e^{-i\theta} \\
    &= r\left(\cos\theta + i\sin \theta \right) + \frac{1}{r}\left( \cos \theta - i\sin\theta\right) \\
    &= \left( r + \frac{1}{r} \right) \cos \theta + i \left(r - \frac{1}{r}\right) \sin \theta.\end{align*}
    
    This is simply the polar form for an ellipse! Thus, we see that the circle $\{|z| = r\}$ is mapped onto an ellipse with equation 
    \[ \frac{u^2}{\left(r + \frac{1}{r}\right)^2} + \frac{v^2}{\left(r - \frac{1}{r}\right)^2} = 1\]

    where $u, v$ represent the real and imaginary components of the images of $z = x+yi$ under $f(z)$. \\

    Observe that since 
    \[ f(z) = re^{i\theta} + \frac{1}{r}e^{-i\theta} = \frac{1}{r} e^{i\theta} + \frac{1}{\left(\frac{1}{r}\right)}e^{i\theta} \]
    we can conclude that $\{|z| = \frac{1}{r} \}$ maps onto the same ellipse, as expected.  \end{solution}

\item \textbf{Show that $f(z)$ is one-to-one on the exterior domain $D = \{|z| > 1\}.$} 
\begin{solution} Let $z_1$ be a solution to $f(z) = w$, where $z_1 \in D$ so $|z_1| > 1$. From part (b), we found that the roots $z_1, z_2$ of the equation $f(z) = w$ must satisfy $z_1z_2 = 1$, or equivalently, $z_1 = \frac{1}{z_2}$. \\

Since $z_1, z_2$ are reciprocals of each other, the other root to $f(z) - w = 0$ must have magnitude 
\[ |z_2| = \frac{1}{|z_1|} < 1\] 
as $|z_1| > 1$. Thus, we know that the other root $z_2$ must lie in the interior domain, and so $f(z)$ is one-to-one on the exterior domain $D$, as desired. \end{solution}

\item \textbf{Determine the image of $D$ under $f(z)$. Sketch the images under $f(z)$ of the circles $\{|z| = r\}$ for $r < 1$, and sketch also the images of the parts of the rays $\{\mathrm{arg} \, z = \beta \}$ lying in $D$.}

\begin{solution} From part (c), we found that for $r > 1$, $f(z)$ maps the circles $\{|z| = r\}$ and $\{|z| = \frac{1}{r}\}$ onto the same ellipse. Consequently, the image of $D$ under $f(z)$ is the image of $\{|z| = r\}$ such that $r \neq 1$. \\

The image of $\{|z| = 1\}$ under $f(z)$ is $f(z) = 2\cos \theta$ where $\theta$ is the angle of $z$ with respect to the positive $x$-axis; so $f(z)$ maps points satisfying $|z| = 1$ to the $[-2, 2]$ portion on the real axis. 
The image of $D$ under $f(z)$ is all other points, namely $\boxed{\C \setminus [-2, 2]}.$  \\

Finally, as we determined in part (c), the images of $f(z)$ of the circles $\{|z| = r\}$ for $r < 1$ correspond to ellipses, which are more stretched out in the $x$-direction at first but approach a circle as $r$ decreases (since $|r+1/r| > |r-1/r|$ but by less and less as $r \rightarrow 0$). 
The rays $\mathrm{arg} \, z = \beta$ lying in $D$ intersect circles $\{|z| = r\}$ orthogonally and thus their images under $f(z)$ correspond to parabolas (due to the ``limiting" nature of the ellipses approaching a circle) intersecting the resulting ellipses orthogonally.
\begin{center}
\includegraphics*[scale = 0.2]{II.6.6.jpeg}
\end{center}

\end{solution}
\end{enumerate}
\newpage

\underline{Section II.7 (Fractional Linear Transformations), II.7.6} \\

\textbf{Show that the image of a straight line under the inversion $z \mapsto 1/z$ is a straight line or circle, depending on whether the line passes through the origin.} \\

\begin{solution}
By definition, any straight line is defined by $Ax + By = C$ for some complex numbers $A, B, C$. We will show that if $C \neq 0$ (meaning the line does not through the origin), the image of this line under inversion is a circle, and if $C = 0$ (meaning the line passes through the origin), the image of this line under inversion is a line. \\

Let $z = x + iy$ be any point on our original straight line \[Ax + By = C.\] Dividing both sides of this equation by $x^2+y^2$, we know that the line is similarly defined by the equation
\[ \frac{Ax}{x^2+y^2} + \frac{By}{x^2+y^2} = \frac{C}{x^2+y^2}. \]
By definition, the image of this point under inversion is some point $z'$ where
\[ z' = \frac{1}{x+iy} = \frac{1}{x+iy} \cdot \frac{x-iy}{x-iy} = \frac{x-iy}{x^2+y^2} = \frac{x}{x^2+y^2} - \frac{y}{x^2+y^2}i. \] 


We see that the result of $z$ after inversion is some point $z' = u + iv$, where $u = \frac{x}{x^2+y^2}$ and $v = -\frac{y}{x^2+y^2}$, and \[u^2 + v^2 = \left( \frac{x}{x^2+y^2} \right)^2 + \left(-\frac{y}{x^2+y^2} \right)^2 = \frac{1}{x^2+y^2}. \]

Substituting each of these expressions in to the equation of our straight line $\frac{Ax}{x^2+y^2} + \frac{By}{x^2+y^2} = \frac{C}{x^2+y^2}$, we get the equation
\[ Au - Bv = C(u^2+v^2).\]

By inspection, if $C=0$, then we get $Au - Bv = 0$, which is simply the equation of a straight line. Consequently, we conclude that if a line passes through the origin, its image under inversion is still a straight line. \\

On the other hand, if $C \neq 0$, let us move all the terms to one side and group related terms to get
\[ (Cu^2 - Au) + (Cv^2 + Bv) = 0.\]

Completing the square, we find that
\[ C\left(u - \frac{A}{2C}\right)^2 + C\left(v + \frac{B}{2C}\right)^2 = \frac{A^2 + B^2}{4C^2}.\]

When $C \neq 0$, this is simply the equation of a circle, centered at $\frac{A}{2C} - \frac{B}{2C} i$ with radius $\frac{1}{2C}\sqrt{A^2 +B^2}.$ Consequently, we conclude that if a line does not pass through the origin, its image under inversion is a circle. \\

Combining our two cases together, we find that the image of a straight line under the inversion $z \mapsto 1/z$ is a straight line or circle, depending on whether the line passes through the origin, as desired.
\end{solution}
\end{document}
